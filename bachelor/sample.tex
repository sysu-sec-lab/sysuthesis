% vim:ts=4:sw=4
%
% Copyright (c) 2008-2009 solvethis
% Copyright (c) 2010-2012 Casper Ti. Vector
% Public domain.
%
% 使用前请先仔细阅读 pkuthss 和 biblatex-caspervector 的文档,
% 特别是其中的 FAQ 部分和用红色强调的部分。
% 两者可在终端/命令提示符中用
%   texdoc pkuthss
%   texdoc biblatex-caspervector
% 调出。

% 在黑白打印时彩色链接可能变成浅灰色,此时可将“colorlinks”改为“nocolorlinks”。
\documentclass[UTF8, colorlinks, notocbibind]{pkuthss}

% 使用 biblatex 排版参考文献,并规定其格式。
\usepackage[backend = biber, style = sysu, utf8]{biblatex}
% 使得打字机粗体可以被使用。
\usepackage{lmodern}
% 产生 originauth.tex 里的 \square。
\usepackage{amssymb}
% 提供 Verbatim 环境和 \VerbatimInput 命令。
\usepackage{fancyvrb}
% 图像包
\usepackage{subcaption}
% --引入其他pdf文件--
\usepackage{pdfpages}

% 使被强调的内容为红色。
\newcommand{\myemph}[1]{\emph{\textcolor{red}{#1}}}

% pkuthss 文档模版的版本。
\newcommand{\docversion}{v1.4 rc1}

% 设定文档的基本信息。
\pkuthssinfo{
	cthesisname = {本科生毕业论文}, ethesisname = {Undergraduate Thesis},
	ctitle = {中山大学论文文档模版\\pkuthss \docversion},
	% “\\”在设定 pdf 属性时会被自动过滤掉,于是得到的 pdf 属性中标题为
	%   The PKU dissertation document classpkuthss [版本号]
	% 此处指定其被替换为“: ”,以使之为
	%   The PKU dissertation document class: pkuthss [版本号]
	etitle = {%
		The PKU dissertation document class\texorpdfstring{\\}{: }%
		pkuthss \docversion%
	},
	cauthor = {罗尼 $\cdot$ 奥沙利文},
	eauthor = {Ronnie O'Sullivan},
	studentid = {07302???},
	date = {二〇一一年五月},
	school = {信息科学与技术学院},
	cmajor = {计算机科学与技术}, emajor = {Computer Science and Technology},
	direction = {强力斯诺克},
	cmentor = {XX 教授}, ementor = {Prof.\ XX},
	ckeywords = {\LaTeX2e{},排版,文档类,\CTeX{}},
	ekeywords = {\LaTeX2e{}, typesetting, document class, \CTeX{}}
}
% 导入参考文献数据库(注意不要省略“.bib”)。
\addbibresource{sample.bib}

%\includeonly{chap/chap1, chap/chap2}

\begin{document}
	% 以下为正文之前的部分。
	\frontmatter

	% 自动生成标题页。
	\maketitle
	% --可以使用includepdf直接将转换好的附表或封面加入--
	%\includepdf[pages=1-2]{chap/cover.pdf}
	%\includepdf[pages=1-2]{chap/report.pdf}
	%\includepdf[pages=1-2]{chap/check.pdf}
	%\includepdf[pages=1-2]{chap/reply.pdf}
	%\includepdf[pages=1-2]{chap/statement.pdf}
	% 版权声明。
	% vim:ts=4:sw=4
%
% Copyright (c) 2008-2009 solvethis
% Copyright (c) 2010-2012 Casper Ti. Vector
% All rights reserved.
%
% Redistribution and use in source and binary forms, with or without
% modification, are permitted provided that the following conditions are
% met:
%
% * Redistributions of source code must retain the above copyright notice,
%   this list of conditions and the following disclaimer.
% * Redistributions in binary form must reproduce the above copyright
%   notice, this list of conditions and the following disclaimer in the
%   documentation and/or other materials provided with the distribution.
% * Neither the name of Peking University nor the names of its contributors
%   may be used to endorse or promote products derived from this software
%   without specific prior written permission.
% 
% THIS SOFTWARE IS PROVIDED BY THE COPYRIGHT HOLDERS AND CONTRIBUTORS "AS
% IS" AND ANY EXPRESS OR IMPLIED WARRANTIES, INCLUDING, BUT NOT LIMITED TO,
% THE IMPLIED WARRANTIES OF MERCHANTABILITY AND FITNESS FOR A PARTICULAR
% PURPOSE ARE DISCLAIMED. IN NO EVENT SHALL THE COPYRIGHT HOLDER OR
% CONTRIBUTORS BE LIABLE FOR ANY DIRECT, INDIRECT, INCIDENTAL, SPECIAL,
% EXEMPLARY, OR CONSEQUENTIAL DAMAGES (INCLUDING, BUT NOT LIMITED TO,
% PROCUREMENT OF SUBSTITUTE GOODS OR SERVICES; LOSS OF USE, DATA, OR
% PROFITS; OR BUSINESS INTERRUPTION) HOWEVER CAUSED AND ON ANY THEORY OF
% LIABILITY, WHETHER IN CONTRACT, STRICT LIABILITY, OR TORT (INCLUDING
% NEGLIGENCE OR OTHERWISE) ARISING IN ANY WAY OUT OF THE USE OF THIS
% SOFTWARE, EVEN IF ADVISED OF THE POSSIBILITY OF SUCH DAMAGE.

\chapter*{版权声明}
{
	\zihao{3}\linespread{1.5}\selectfont

\iffalse
	任何收存和保管本论文各种版本的单位和个人,
	未经本论文作者同意,不得将本论文转借他人,
	亦不得随意复制、抄录、拍照或以任何方式传播。
	否则一旦引起有碍作者著作权之问题,将可能承担法律责任。
	\par
\fi
	版权所有 \copyright\ 2008--2009 solvethis
	\par
	版权所有 \copyright\ 2010--2012 Casper Ti. Vector
	\vskip 1em

	pkuthss 文档类和 pkuthss-extra 宏包以 %
	\LaTeX{} Project Public License 发布。
	本说明(示例)文档的源代码中,除以下文件
	\begin{itemize}
		\item \verb|img/sysulogo.eps|
		\item \verb|chap/copyright.tex|
		\item \verb|chap/originauth.tex|
	\end{itemize}
	以 New BSD License 发布,
	以及参考文献和引用使用的样式文件版权声明见相应文件中说明以外,
	其余部分文件发布在公有领域(public domain)。
	\par
}


	% 中英文摘要。
	\include{chap/abstract}
	% 自动生成目录。
	\tableofcontents

	% 以下为正文。
	\mainmatter

	% 绪言。
	\include{chap/introduction}
	% 各章节。
	\include{chap/chap1}
	\include{chap/chap2}
	\include{chap/chap3}
	% 结论。
	\include{chap/conclusion}

	% 正文中的附录部分。
	\appendix
	% 排版参考文献列表,并使其出现在目录中。
	% 如果同时要使参考文献列表参与章节编号,可将“bibintoc”改为“bibnumbered”。
	\printbibliography[heading = bibintoc]
	% 各附录。
	\include{chap/encl1}
	% vim:ts=4:sw=4
%
% Copyright (c) 2008-2009 solvethis
% Copyright (c) 2010-2012 Casper Ti. Vector
% Public domain.

\chapter{更新记录}
\raggedbottom % 避免某些奇怪的“Underfull \vbox”警告。

\section{1.3 版以后的更新记录}
\VerbatimInput[
	tabsize = 4, fontsize = {\small}, baselinestretch = 1
]{./verbatim/ChangeLog.txt}

\section{1.3 及其以前版本的更新记录}
\VerbatimInput[
	tabsize = 4, fontsize = {\small}, baselinestretch = 1.1
]{./verbatim/ChangeLog-upto-1.3.txt}

\flushbottom % 取消 \raggedbottom 的作用。



	% 以下为正文之后的部分。
	\backmatter

	% 致谢。
	\include{chap/acknowledge}
	% 原创性声明和使用授权说明。
	\include{chap/originauth}
	% 原创性声明和使用授权说明,不显示页码。
	%\pagestyle{empty}
	%\include{chap/originauth}
	%--成绩评定表--
	% \includepdf[pages=1-2]{chap/grade.pdf}
\end{document}

